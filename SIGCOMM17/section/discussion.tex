\section{Discussions}
\label{sec:discussion}

We point out a few limitations of \nfactor, and intriguing future directions.

%To achieve transparent resilience,
To achieve clean separation between flow state and NF core logic, \nfactor~requires NFs rewritten using a set of APIs (Sec.~\ref{sec:NFAPIs}). With the advance of NFV, more and more new NFs will be created. If using the actor model for constructing NFV systems is accepted by the community, we believe it feasible to create new NFs following our design. %, that processes flows based on flow state could achieve transparent resilient if they are implemented using~\nfactor.

Though focusing on stateful NFs, %Due to the use of lightweight actors, 
\nfactor~can handle non-stateful NFs as well with ease. Non-stateful NFs can also benefit from the fast, distributed flow migration, as it eliminates potential packet re-ordering caused by directly changing the path of a flow.
%\chuan{briefly discuss what if nfactor is used to handle non-stateful NFs, short flows}

%Even though~\nfactor provides transparent resilience for stateful NFs,
\nfactor~focuses on handling per-flow state, consisting of states of NFs along the flow path. The current \nfactor~framework does not correctly handle shared states, \ie, the states shared by a bunch of flows \cite{bro}. The reason is that %Even though the NF API in~\nfactor achieves a clean separation between per-flow state and NF processing logic,
our current NF API design does not achieve correct separation of shared states. Migrating and replicating flows that share states with other flows may lead to un-predictable errors. A potential solution is to implement a handler on flow actor that explicitly deals with state inconsistency during flow migration and replication. We leave this to our future work.

In addition, \nfactor~may incorrectly handle flows with packet encapsulation. \nfactor~uses the 5-tuple to differentiate flows. Different flows may share the same 5-tuple if their original flow packets are encapsulated using similar headers. This is common for flows sent over the same VxLAN tunnel. In that case, those flows are handled by the same flow actor using the same service chain, potentially incorrect. If we know what kind of encapsulation the input flows use, we may add a decapsulation function in the virtual switch to correctly extract different flows. This will also be further investigated in our future work.
